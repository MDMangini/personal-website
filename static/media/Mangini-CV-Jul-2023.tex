% !TeX program = XeLaTeX

%% LyX 2.1.4 created this file.  For more info, see http://www.lyx.org/.
%% Do not edit unless you really know what you are doing.
\documentclass[english]{simplecv}
\usepackage[T1]{fontenc}
\usepackage{fontspec}
%\usepackage[latin9]{inputenc}
\setmainfont{Crimson Text}
\setsansfont{Oxygen}
\usepackage{geometry}
\geometry{verbose,lmargin=1in,rmargin=1in}
\setcounter{secnumdepth}{2}
\setcounter{tocdepth}{2}
\usepackage{babel}
\usepackage{url}
\begin{document}

\leftheader{477 Prospect St \\ APT 1L \\ New Haven, CT, 06511}


\rightheader{michael-david.mangini@yale.edu\\
(T) (203) 788-5545\\
\texttt{michaeldavidmangini.com}}


\title{Michael-David Mangini}

\maketitle
\begin{center}
Postdoctoral Fellow at the Leitner Program in Comparative and International Political Economy
\par\end{center}

\begin{center}
Yale University
\par\end{center}

\section{Appointments}
\begin{topic}
\item 2022 Postdoctoral Fellow at the Niehaus Center for Globalization and Governance
\end{topic}

\section{Education}
\begin{topic}
\item Ph.D. 2022 in Political Economy and Government\\
Harvard University

\item M.A. 2018 in Political Economy and Government\\
Harvard University

\item B.A. May 2014 \emph{summa cum laude} in Economics and International Relations\\
University of Pennsylvania
\end{topic}

\section{Publications}

\begin{topic}
\item 
Mangini, Michael‐David. \textbf{``Escape from Tariffs: The Political Economies of Protection and Classification.''} \emph{Economics \& Politics} (2022). \url{https://doi.org/10.1111/ecpo.12244}  \\\\
Abstract: The literature on the political economy of trade protection has focused on how firms lobby for their preferred tariffs, but opportunities to change the tariff schedule in the United States by legislation are relatively uncommon. How can importers lobbying in their private capacity escape from tariffs without relying on Congress? In the United States, products must be sorted into tariff categories by the customs office before the appropriate duty can be determined. This article documents how firms seek to lower their tariffs by strategically requesting product classifications from customs. Firms hire lawyers to make legal arguments to customs officials promoting an interpretation of the tariff schedule which lowers their costs. Lobbying for favorable classification is most important to the firm when similar categories have very different tariffs. The language of product descriptions is political because legal arguments for a particular classification are more persuasive when the descriptions are worded flexibly. Using a data set of over $200,000$ classification rulings between 1990 and 2020, I find evidence that firms request classifications in response to certain quotas and the China tariffs. The findings characterize the tariff schedule as a living document and describe how the distribution of tariffs and the language of product descriptions affect the structure of protection.
\end{topic}

\begin{topic}
\item 
Mangini, Michael-David. \textbf{``The Economic Coercion Trilemma''}. \emph{Journal of Conflict Resolution} (2023). \url{https://doi.org/10.1177/00220027231191530} \\\\
Abstract: States often use market access as a bargaining chip in international politics. A state that requires simultaneous compliance in multiple issue areas before granting market access maximizes incentives to comply but also makes them brittle -- any targeted states that cannot comply in one issue area have no incentive to comply in any. More generally, programs of economic coercion can achieve at most two of the following three objectives: 1) secure a broad coalition of domestic political support, 2) the association of meaningful trade value with each policy issue, and 3) assurance that enforcing one political issue will not reduce the target's incentives to comply with conditionality on others. Characteristics of the program's domestic constituency, of the issues themselves, and of the international economy are key determinants of how the state prioritizes the three objectives. The trilemma explains the number and types of issues that can be linked to economic value.
\end{topic}


\section{Working papers}

\begin{topic}
\item \textbf{Coalitions and the Politics of Restraint: Evidence from the Iran Deal Negotiation} \\
\emph{Under Review} \\
Presented at Harvard International Relations and Political Economy workshops \\
Abstract: Economic coercion depends on the credibility of both threats to punish noncompliance and assurances that compliance will not be punished. What instruments can states deploy to make the necessary assurances without undermining the credibility of their threats? This article describes how some factors that bolster the credibility of threats can simultaneously undermine the credibility of assurances. It then argues that states can mitigate the challenge by carefully selecting coalition partners with different interests who can hold them accountable. The paper applies the theory to the Iran deal negotiation and finds that Congressional resolve to maintain sanctions initially stymied progress. The United States was ultimately able to increase the believability of its commitments by partnering with European states that were more open to removing sanctions.
\end{topic}

\begin{topic}
\item \textbf{How Effective is Trade Conditionality? Economic Coercion in the Generalized System of Preferences} \\
\emph{Under Review} \\ 
Abstract: The Generalized System of Preferences (GSP) exemplifies the ways in which international economic linkages can become conduits of political influence. The program offers beneficiary developing countries the opportunity to export a wide variety of goods duty free to the United States, but eligibility is conditional on labor and intellectual property rights protections. How effective are programs like the GSP at causing states to change behavior by raising the risk that market access will be revoked? The challenge of detecting economic coercion in programs like the GSP is identifying the program's influence in states where the conditionality is never enforced. The paper applies a new conceptual approach to show that GSP beneficiaries change policy to reduce the risk of being threatened with expulsion from the program. An implication of the findings is that the political consequences of economic linkages could be far more widespread than previously thought.
\end{topic}

\begin{topic}
\item \textbf{Robots, Foreigners, and Foreign Robots: Policy Responses to Automation and Trade} (with Stephen Chaudoin) \\
\emph{Under Review} \\
Presented at GRIPE in August 2021, IPES in October 2022 \\
Abstract: Why do politicians blame offshoring for job losses and not automation? Why do voters demand tariffs and not redistribution? We argue that the collision of economic nationalism and comparative advantage answers both. Opportunistic politicians emphasize offshoring because economic nationalist voters - who dislike imports – support tariffs but not automation regulations. We develop a general formal model of a citizen’s demand for policy in response to economic shocks, where citizens form preferences over redistribution and a policy that blunts the shock. The source (foreign versus domestic) and type (labor versus automation) of a shock affects the citizen’s preferred policy bundle. We use survey experimental evidence to show that domestic automation shocks increase relative support for redistribution versus regulations, while globalization increases weight on protectionism. Emphasizing foreign-produced automation reweights responses towards regulations. This enhances our understanding of anti-globalization sentiment and explains how the tide could turn against automation.
\end{topic}

\begin{topic}
\item \textbf{Conflict Technology as a Catalyst of State Formation: Urban Fortifications in Medieval and Early Modern Europe} (with Casey Petroff)  \\
\emph{Under Review} \\ 
\emph{Honorable Mention for Best Poster at the Harvard Government Department Poster Session} \\
We argue that the gunpowder revolution in medieval Europe encouraged the amalgamation of smaller polities into larger centralized states. The shock to military technology made existing fortifications obsolete and dramatically raised the cost of defensive investments. Small polities lacked the fiscal capacity to make these investments, so they had either to ally or merge with others. Alliances created prospects of free-riding by interior cities on border cities. In contrast, unitary centralized states benefited from geographic and fiscal economies of scale, facilitating defensive investments at the border that protected the interior while limiting free-riding and resource misallocation. Using a new dataset on fortifications in over 6,000 European cities, we find that states made defensive investments in areas of territorial contestation, closer to borders, and farther from raw building materials. These findings are consistent with the theory that large centralized states arose in part as a consequence of changes in military technology.
\end{topic}

% \section{Invited Talks}
% \begin{topic}
% \item [{2020}] American Political Science Association 
% \item [{2020}] International Political Economy Society 
% \item [{2018-2020}] Seminar Presentations at Harvard
% \end{topic}

\section{Honors and Awards}
\begin{topic}
\item [{2020}] Graduate Student Associate of the Weatherhead Center for International Affairs
\item [{2018-2020}] Graduate Student Associate of the Institute for Quantitative Social Science
\item [{2019}] Harvard Kennedy School Distinction in Teaching Award (for masters level microeconomics)
\item [{2018}] Harvard Distinction in Teaching Award (for PhD level microeconomics)
\item [{2014}] Joseph Warner Yardley Prize \\
The Yardley Prize is awarded to the best thesis on political economy
by a senior in any undergraduate school at Penn. Awarded for the
thesis titled, \textquotedblleft A Tale of Two Controversies: Impact
of IMF Bias on Moral Hazard 1990-2010.\textquotedblright{} 
\item [{2014}] Erik Arnetz Integrating Knowledge Award \\
 Awarded to an International Relations major whose \textquotedblleft ingenuity,
initiative, creativity, and academic achievement exemplify the values
of connecting research to practice throughout the entirety of their
classroom study, internship and work experience, research opportunities,
and thesis writing.\textquotedblright{} 
\item [{2014}] Phi Beta Kappa inductee 
\item [{2010}] Benjamin Franklin Scholars Program
\item [{2012}] Special Delegate to West Point SCUSA Conference on International Affairs and Public Policy 
\end{topic}

\section{Professional Experience}
\begin{topic}
\item[{Aug 2017 - Present}] Teaching Fellow at the Harvard Kennedy School and the Harvard Faculty of Arts and Sciences

\begin{itemize}
\item BGP610: The Political Economy of Trade (for Professor Robert Z. Lawrence). Masters of Public Policy and Masters of Public Administration students.
\item Econ1400: The Future of Globalization (for Professors Robert Z. Lawrence and Lawrence H. Summers). Undergraduate students.
\item Gov40: Introduction to International Relations (for Professors Dustin Tingley and Stephen Chaudoin). Undergraduate students.
\item Econ2020a: Microeconomic Theory (for Professor Maciej Kotowski). PhD students.
\item Gov1780: International Political Economy (for Professor Jeffry Frieden). Undergraduate students.
\item API101: Markets and Market Failure (for Professors Pinar Dogan, David Ellwood, Janina Matuszeski, and Marcella Alsan). Masters of Public Policy students.
\end{itemize}

\end{topic}


\begin{topic}
\item [{March 2019 - July 2019}] Research Assistant to Professors Josh Kertzer and Marcus Holmes
\begin{itemize}
\item Reviewed the political economy literature on the problem of conceptualizing the aggregation of preferences and biases of individuals. 
\end{itemize}
\end{topic}


\begin{topic}
\item [{July 2016 - Feb 2017}] Research Assistant to Professor Dani Rodrik
\begin{itemize}
\item Assisted the construction and analysis of an original dataset on the rise of populism and its connection to globalization. 
\end{itemize}
\end{topic}


\begin{topic}
\item [{Mar~2014~-~May~2015}] Bates White Economic Consulting\\
\emph{Consultant}
\begin{itemize}
\item Analyzed economic datasets using econometric techniques and briefed senior consultants on the results. 
\item Conducted damages and penalties analysis with an expert in corporate litigation involving pharmaceutical pricing and marketing using Stata and SQL. 
\item Supported an expert in analyzing the economics of blood plasma donation. 
\item Contributed to expert witness deposition preparation.
 \end{itemize}
\end{topic}
%\begin{topic}
%\item [{Oct~2012~-~May~2013}] PLANETRONIX/Meidlinger Partners LLC \\
%\emph{ Research Assistant in Private Equity }\end{topic}
%\begin{itemize}
%\item Created a revenue model for Planetronix, an entrepreneurial startup, to demonstrate revenue potential 
%\item Provided significant contributions to business development strategy for startup enterprise and conducted research on related competitive
%industries to identify business opportunity model, including options
%for funding 
%\end{itemize}

\begin{topic}
\item [{Jun~2011~-~Feb~2013}] International Assessment and Strategy
Center\\
\emph{Research Assistant to Professor Anne-Louise Antonoff }
\begin{itemize}
\item Project 2030: An analysis of international trends through the year
2030 in critical infrastructure and resource management. Studied cyber-security,
rare earth metals, international finance trends, energy technology,
and many other trends for a major report on issues and concerns anticipating
2030. 
\item Transnational Organized Crime: Researched and wrote reports on various
aspects of transnational organized crime. Specifically, the team studied
the crime-terror nexus and the relationship between ungoverned physical
spaces and crime. The territories of Mexico, Russia, Italy, Afghanistan,
Ghana, Hong Kong, China, Taiwan, and others formed the foundation
for case studies. 
\end{itemize}
\end{topic}
\section{General Skills}
\begin{itemize}
\item Software:  \texttt{R}, \LaTeX, Qualtrics, Stata, MATLAB
\item Language: Italian
\end{itemize}

\end{document}
